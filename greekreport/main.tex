\documentclass[a4paper]{article}
\usepackage[utf8]{inputenc}

%% Greek and English text}
%% based on http://tex.stackexchange.com/questions/110573/writing-greek-text
\usepackage[english,greek]{babel}


\title{Διαχείριση Μεγάλων Δεδομένων}
\author{Παρασκευή Ραυτοπούλου \\
Τμήμα Πληροφορικής και Τηλεπικοινωνιών \\
Πανεπιστήμιο Πελοποννήσου}
\date{Νοέμβριος 2018}

\usepackage{natbib}
\usepackage{graphicx}

\begin{document}

\maketitle

\begin{abstract}
Παρουσιάζουμε το \selectlanguage{english} i\textsc{Cluster}, \selectlanguage{greek} ένα καταπληκτικό σύστημα για διαχείριση μεγάλων δεδομένων. Το θέμα μας ήταν η συμμετρία, θα παρουσιάσουμε γνωστά συστήματα στο χώρο της διαχείρισης μεγάλων δεδομένων και θα δείξουμε πώς το \selectlanguage{english} i\textsc{Cluster} \selectlanguage{greek} διαφοροποιείται και κρατά την πρώτη θέση στη συμμετρική κατανομή των δεδομένων σε ένα μεγάλης εμβέλειας \selectlanguage{english} cloud.   
\end{abstract}

\selectlanguage{english}{\bf Keywords:} enter, keyword, here.

\selectlanguage{greek}
\section{Introduction}
There is a \textbf{theory} which states that \textit{if ever anyone discovers exactly what the Universe is for and why it is here, it will instantly disappear and be replaced by something even more bizarre and inexplicable} \cite{miller90introduction}. For more details look at the Figure \ref{fig:universe}.

There is another \textbf{theory} which states that \textit{this has already happened} \cite{aberer03chatty}. The authors present the stars, show a nice picture of the moon, and laugh out loud because they drink a lot of alcohol.


\begin{figure}[h!]
\centering
\includegraphics[scale=1.7]{universe}
\caption{The Universe}
\label{fig:universe}
\end{figure}
\section{Σχετικές εργασίες} \label{sec:related-works}
Διάβασα πολλά άρθρα, εδώ θα σας τα παρουσιάσω με λεπτομέρειες και κριτική ματιά...

Αυτό το άρθρο \cite{adams1995hitchhiker} είχε μια πολύ ωραία εξίσωση που εκτείνεται σε πολλές γραμμές, δες παρακάτω.
\begin{equation}\label{eq:weighted-coef} 
w\gamma_i=\frac{1}{s_i(s_i - 1)}\sum{S(p_i,p_k)}, \ \ p_j,p_k \in RI_i,\ p_k \in RI_j
\end{equation}

Μπορείς πάντα βέβαια να βάζεις μαθηματικά ή σύμβολα και μέσα στο κείμενο με τον τρόπο αυτό $\theta$, $t_F$, $t_q$, $\varrho$, $s_i$ και $l_i$ και να δείχνουν μια χαρά, μόνο που σε αυτήν την περίπτωση δε μπορείς να χρησιμοποιήσεις \selectlanguage{english}label \selectlanguage{greek}και να αναφερθείς σε αυτά αλλού στο κείμενό σου. 

Επίσης, μπορείς στο κείμενό σου να βάζεις λίστες με διαφορετικούς τρόπους ανάλογα με τις ανάγκες σου. Για παράδειγμα, μια λίστα φαίνεται παρακάτω.

\begin{description}
    \item[πρώτο αντικείμενο] περιγραφή πρώτου αντικειμένου
    \item[δεύτερο αντικείμενο] περιγραφή δεύτερου αντικειμένου
\end{description}

Βέβαια, οι πιο συνηθισμένες λίστες είναι αυτές με αρίθμηση, όπως η επόμενη, όπου φυσικά μπορείς να έχεις οσαδήποτε αντικείμενα, οι αριθμοί μπαίνουν αυτόματα.

\begin{enumerate}
    \item Περιγραφή πρώτου αντικειμένου
    \item Περιγραφή δεύτερου αντικειμένου
    \item Περιγραφή τρίτου αντικειμένου
\end{enumerate}

Ή ακόμα κι αυτές με  τελείες, όπως η επόμενη. Υπάρχει φυσικά κι η δυνατότητα να έχεις υπολίστα μέσα στη λίστα ακολουθώντας την ίδια λογική. 

\begin{itemize}
    \item Περιγραφή πρώτου αντικειμένου, ακολουθείται από μια υπολίστα
    \begin{itemize}
        \item Πρώτη υποπερίπτωση
        \item Δεύτερη υποπερίπτωση
        \item κοκ.
    \end{itemize}
    \item Περιγραφή δεύτερου αντικειμένου
    \item Περιγραφή τρίτου αντικειμένου
    \item Και τέλος πάντων, προσθέτει όσα αντικείμενα χρειάζεσαι.
\end{itemize}

Σε όλες τις παραπάνω περιπτώσεις λίστας μπορείς να χρησιμοποιήσεις \selectlanguage{english}labels \selectlanguage{greek}αν χρειάζεται να αναφερθείς σε αυτή κάπου αλλού στο κείμενό σου, αν και δε συνηθίζεται. Τα l\selectlanguage{english}abels \selectlanguage{greek}μπαίνουν με τον κλασικό τρόπο, δες για παράδειγμα παραπάνω στο κεφάλαιο την Εξίσωση \ref{eq:weighted-coef}. 

\section{Συμπεράσματα} \label{sec:conclusions}
Τελικά αυτό το μάθημα έχει πολύ μεγάλο ενδιαφέρον! 

\bibliographystyle{plain}
\selectlanguage{english}
\bibliography{references}
\end{document}
