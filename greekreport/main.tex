\documentclass[a4paper]{article}
\usepackage[utf8]{inputenc}

%% Greek and English text}
%% based on http://tex.stackexchange.com/questions/110573/writing-greek-text
\usepackage[english,greek]{babel}


\title{Διαχείριση Μεγάλων Δεδομένων}
\author{Παρασκευή Ραυτοπούλου \\
Τμήμα Πληροφορικής και Τηλεπικοινωνιών \\
Πανεπιστήμιο Πελοποννήσου}
\date{Νοέμβριος 2018}

\usepackage{natbib}
\usepackage{graphicx}

\begin{document}

\maketitle

\begin{abstract}
Παρουσιάζουμε το \selectlanguage{english} i\textsc{Cluster}, \selectlanguage{greek} ένα καταπληκτικό σύστημα για διαχείριση μεγάλων δεδομένων. Το θέμα μας ήταν η συμμετρία, θα παρουσιάσουμε γνωστά συστήματα στο χώρο της διαχείρισης μεγάλων δεδομένων και θα δείξουμε πώς το \selectlanguage{english} i\textsc{Cluster} \selectlanguage{greek} διαφοροποιείται και κρατά την πρώτη θέση στη συμμετρική κατανομή των δεδομένων σε ένα μεγάλης εμβέλειας \selectlanguage{english} cloud.   
\end{abstract}

\selectlanguage{english}{\bf Keywords:} enter, keyword, here.

\selectlanguage{greek}
\section{Εισαγωγή}
Αυτό η αναφορά αποτελεί την εργασία μου για το μάθημα ``Διαχείριση Μεγάλων Δεδομένων''. Παρακάτω θα παρουσιάσω αναλυτικά τα θέματα με τα οποία ασχολήθηκα, τα άρθρα που διάβασα σχετικά \cite{adams1995hitchhiker}, και τα συμπεράσματα μου.

Για να παρουσιάσουμε το θέμα, πρώτα θα συζητήσουμε τις διαφορετικές χρήσεις κι εφαρμογές των μεγάλων δεδομένων. Τα μεγάλα δεδομένα χρησιμοποιούνται σε διάφορους τομείς\footnote{Θα αναφέρουμε μόνο τους 10 πιο δημοφιλείς σύμφωνα με το \selectlanguage{english} https://www.smartdatacollective.com/big-data-analytics-create-billion-dollar-mobile-app-ux/.} \selectlanguage{greek} της ζωής μας.
Μπλα μπλα μπλα (όπως φαίνεται στον Πίνακα \ref{tab:classes}), όπου φαίνονται τα επιμέρους χαρακτηριστικά των εφαρμογών.

\begin{table}[h]
\centering \caption{Χαρακτηριστικά ανά εφαρμογή}
\begin{tabular}{|l||c|}
  \hline
  % after \\: \hline or \cline{col1-col2} \cline{col3-col4} ...
  \textbf{Εφαρμογή} & \textbf{Χαρακτηριστικά} \\
  \hline \hline
  \selectlanguage{english}antibodies & 3400 \\
  \selectlanguage{english}DNA & 2122 \\
  \selectlanguage{english}molecular sequence data & 3426 \\
  \selectlanguage{english}molecular sequence data & 3426 \\
  \selectlanguage{english}pregnancy & 4658 \\
  \selectlanguage{english}prognosis & 2715 \\
  \selectlanguage{english}receptor & 938 \\
  \selectlanguage{english}risk factors & 3532 \\
  \selectlanguage{english}tomography & 3594 \\
  \selectlanguage{english}in vitro & 3464 \\
  \hline
\end{tabular}
\label{tab:classes}
\end{table}

Ο παραπάνω πίνακας δείχνει ξεκάθαρα ότι η εφαρμογή \selectlanguage{english}DNA \selectlanguage{greek}υπολείπεται των υπολοίπων περίπου 1500 χαρακτηριστικά, πράγμα που την κάνει λιγότερη δημοφιλή ανάμεσα στους χρήστες. Είναι προφανές ότι η πιο δημοφιλής εφαρμογή είναι το \selectlanguage{english} i\textsc{Cluster}, \selectlanguage{greek} το οποίο έχει πολλαπλάσια χαρακτηριστικά έναντι όλων των υπολοίπων.

Στα κεφάλαια που ακολουθούν θα αναλύσουμε το σκεπτικό μας. Συγκεκριμένα, στο Κεφάλαιο \ref{sec:related-works} θα παρουσιάσουμε όλες τις σχετικές εργασίες, στο Κεφάλαιο μπλα μπλα μπλα, και τέλος στο Κεφάλαιο \ref{sec:conclusions} θα παρουσιάσουμε μια συγκριτική μελέτη και τα συμπεράσματά μας.

\section{Σχετικές εργασίες} \label{sec:related-works}
Διάβασα πολλά άρθρα, εδώ θα σας τα παρουσιάσω με λεπτομέρειες και κριτική ματιά...

Αυτό το άρθρο \cite{adams1995hitchhiker} είχε μια πολύ ωραία εξίσωση που εκτείνεται σε πολλές γραμμές, δες παρακάτω.
\begin{equation}\label{eq:weighted-coef} 
w\gamma_i=\frac{1}{s_i(s_i - 1)}\sum{S(p_i,p_k)}, \ \ p_j,p_k \in RI_i,\ p_k \in RI_j
\end{equation}

Μπορείς πάντα βέβαια να βάζεις μαθηματικά ή σύμβολα και μέσα στο κείμενο με τον τρόπο αυτό $\theta$, $t_F$, $t_q$, $\varrho$, $s_i$ και $l_i$ και να δείχνουν μια χαρά, μόνο που σε αυτήν την περίπτωση δε μπορείς να χρησιμοποιήσεις \selectlanguage{english}label \selectlanguage{greek}και να αναφερθείς σε αυτά αλλού στο κείμενό σου. 

Επίσης, μπορείς στο κείμενό σου να βάζεις λίστες με διαφορετικούς τρόπους ανάλογα με τις ανάγκες σου. Για παράδειγμα, μια λίστα φαίνεται παρακάτω.

\begin{description}
    \item[πρώτο αντικείμενο] περιγραφή πρώτου αντικειμένου
    \item[δεύτερο αντικείμενο] περιγραφή δεύτερου αντικειμένου
\end{description}

Βέβαια, οι πιο συνηθισμένες λίστες είναι αυτές με αρίθμηση, όπως η επόμενη, όπου φυσικά μπορείς να έχεις οσαδήποτε αντικείμενα, οι αριθμοί μπαίνουν αυτόματα.

\begin{enumerate}
    \item Περιγραφή πρώτου αντικειμένου
    \item Περιγραφή δεύτερου αντικειμένου
    \item Περιγραφή τρίτου αντικειμένου
\end{enumerate}

Ή ακόμα κι αυτές με  τελείες, όπως η επόμενη. Υπάρχει φυσικά κι η δυνατότητα να έχεις υπολίστα μέσα στη λίστα ακολουθώντας την ίδια λογική. 

\begin{itemize}
    \item Περιγραφή πρώτου αντικειμένου, ακολουθείται από μια υπολίστα
    \begin{itemize}
        \item Πρώτη υποπερίπτωση
        \item Δεύτερη υποπερίπτωση
        \item κοκ.
    \end{itemize}
    \item Περιγραφή δεύτερου αντικειμένου
    \item Περιγραφή τρίτου αντικειμένου
    \item Και τέλος πάντων, προσθέτει όσα αντικείμενα χρειάζεσαι.
\end{itemize}

Σε όλες τις παραπάνω περιπτώσεις λίστας μπορείς να χρησιμοποιήσεις \selectlanguage{english}labels \selectlanguage{greek}αν χρειάζεται να αναφερθείς σε αυτή κάπου αλλού στο κείμενό σου, αν και δε συνηθίζεται. Τα l\selectlanguage{english}abels \selectlanguage{greek}μπαίνουν με τον κλασικό τρόπο, δες για παράδειγμα παραπάνω στο κεφάλαιο την Εξίσωση \ref{eq:weighted-coef}. 

\section{Conclusion}
``I always thought something was fundamentally wrong with the universe'' \citep{adams1995hitchhiker}.

\bibliographystyle{plain}
\selectlanguage{english}
\bibliography{references}
\end{document}
